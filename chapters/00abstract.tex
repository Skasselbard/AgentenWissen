% !TEX root = ../main.tex
\chapter*{Abstract}
Bei vielen agentenbasierten Modellen, insbesondere mit sozialwissenschaftlicher Anwendung, stehen die Entscheidungsprozesse der Agenten im Vordergrund. Zur
Modellierung dieser Entscheidungsprozesse wurde eine Vielzahl von Ansätzen entwickelt\cite{balke2014agents}
Diese Ansätze unterscheiden sich stark in ihrer Komplexität und in der Berucksichtigung
kognitionspsychologischer Phänomene.

In dieser Arbeit soll eine Auswahl an Ansätzen miteinander verglichen werden.
Der Fokus des Vergleichs soll dabei auf dem Wissen der Agenten über ihre Umwelt liegen: Wie kann unvollständiges Wissen und kognitiver Bias modelliert werden?
Dabei ist auch die Umsetzung in konkreten Modellen von Interesse.

\vfill

\begin{tabular}{ll}
	\bfseries Betreuer: & \parbox[t]{10cm}{\betreuer }\vspace{5mm} \\
	\bfseries Tag der Ausgabe: & 28.01.2019 \\
	\bfseries Tag der Abgabe: & 28.02.2019 \\
\end{tabular}
