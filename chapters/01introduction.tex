% !TEX root = ../main.tex
\chapter{Einleitung}
\label{introduction}
Die zentrale Frage die in dieser Arbeit geklärt werden soll ist: "Wie kann man einen Agenten mit unvollständigen Wissen und kognitiver Verzerrung modellieren?". 
Dabei soll die Arbeit einen inhaltlich begrenzten Überblick geben und keine Konzepte im Detail erklären. Es wird sich dabei stark an die Zusammenfassung im Referenz-Paper \cite{balke2014agents} gehalten.

Der Begriff des \enquote{Wissens} ist in diesem Zusammenhang relativ intuitiv als das Bekanntsein von Fakten zu verstehen.
Agenten mit unvollständigen Wissem haben dabei kein komplettes Bild ihrer Umgebung.
Was das genau bedeutet, soll in den nachvolgenden Abschnitten weiterführend erläutert werden.

Vorher soll jedoch noch einmal genauer auf den Begriff der Kognitive Verzerrung eingegangen werden.

\section{Kognitive Verzerrung}
Kognitive Verzerrung ist ein systematische Abweichung von der Norm oder rationalen Einschätzungen\cite{}. 
Es ist ein Phänomen was für gewöhnlich bei Menschen beobachtet wird und äußert sich im Denken und Verhalten der Personen.
Normalerweise ist diese Abweichung dabei unbewusst.

Um kognitive Verzerrung zu beobachten muss zunächst bekannt sein was überhaupt die Norm ist.
Da die empfundene Norm (für Verhalten, Ausprägungen, auftretenden Ereignissen, etc.) jedoch bereits einer Verzerrung unterliegen kann, kann es schwierig werden eine \enquote{tatsächliche Norm} zu definieren. 
Dieses Problem kann mit Wahrscheinlichkeitsmodellen teilweise begrenzt werden. 
So können z.B. unter gewissen Umständen die Wahrscheinlichsten \enquote{Ereignisse} als \enquote{die Norm} angenommen werden.

Üblicherweise ist die kognitive Verzerrung eine hilfreiche Vereinfachung komplexer Umstände und sind damit auch oft mit einer Art von Heuristiken beim denken zu veranschaulichen.

Die Gründe für solche Verzerrungen sind vielfältig.
So können sie wie eben angedeutet z.B. durch die Vereinfachung der Verarbeitung von Informationen oder durch Fehler bei der Erinnrung von Fakten autreten.
Auch die Beeinflussung durch Emotionen kann, wie sozialer Druck, eine Erklärung für gewisse Ausprägungen sein.

Wie sich ein Ansatz für kognitive Verzerrung und das Abbilden von unvollständigen Wissen, in einem der am weitesten verbreiteten Formalismen umsetzen lassen kann, wird im nächsten Abschnit genauer betrachtet.