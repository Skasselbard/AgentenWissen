% !TEX root = ../main.tex


\chapter{Grundlegende Agenten}
Der Begriff des Agenten ist nicht genau definiert, allerdings gibt es verbreitete Zustimmung zu den Grundeigenschaften.
So ist die zentrale Eigenschaft von Software Agenten ihre Autonomie: Agenten finden sich selbständig in ihrer Umgebung bzw. in ihrem System zurecht. Sie können auf Veränderungen in ihrer Umgebung reagieren und verfolgen üblicherweise ein bestimmtes Ziel.[bader]%TODO: Zitat Wooldridge: Intelligent Agents: The Key Concepts. 2002, S. 5.
Agenten eignen sich gut für den Einsatz in verteilten Systemen mit looser Kopplung.
So kann man ein System mit neuen Agenten erweitern indem man einen weiteren Agenten hinzufügt, ohne bestehende Agenten anzupassen und ohne sie anhalten zu müssen.
Durch die Autonomie der Agenten kann außerdem einen hohen Grad an Parralälität erreichen.
Da das Konzept des software Agenten ähnlichkeit mit dem verhalten von Menschen aufweist, ist eine Verwendung zur Simulation und Analyse von menschlichem Verhalten naheliegend.
Die verschiedenen Versuche, menschliche Eigenschaften wie Emotionen und soziale Einflüsse zu modellieren, die im Vergleichspaper zusammengefasst wurden, verdeutlichen dieses Bestreben.

\section{BDI-Agenten}
% weit verbreitet\\
Das Belief-Desire-Intentions (BDI) Modell ist eines der verbreitesten Konzepte um Agenten zu modellieren.

% beliefs desieres intents\\
Die drei namensgebenden Begriffe sind daber die wesentlichen Bestandteile dieses Modells.
Agenten dieser Klasse besitzen eine Wissensbasis (Beliefs) den sie für den aktuellen Zustand ihrer Umgebung halten.
Die Wissensbasis ist z.B. durch Beobachtungen vernänderbar.
Sie muss allerdings nicht dem tatsächlichem "Zustand der Welt" entsprechen.
D.h. dass sie weder vollständig noch korrekt sein muss.
Die Wünsche (Desires) des Agenten sind Ziele die ein Agent gerne erreichen würde.
Sie bestimmen seine Aktionen nicht direkt. 
Erst wenn bestimmte Wünsche in die Vorhaben (Intentions) des Agenten übernommen werden, wird versucht einen Wünsch mit eigenen Aktionen zu erfüllen.
Üblicherweise werden Wissen und Wünsche in Prädikatenlogik formuliert.

Das moddelieren von unvollständigem Wissen kann somit gut durch die "Kontrolle" der Beobachtungen des Agenten erreicht werden. 
So könnte der Agent z.B. ganze Ereignisklassen ignorieren oder die Menge von Beobachtungen zu beschränken (z.B. durch einen virtuellen Ort an dem sich der Agent befindet).
Ohne die Fähigkeit Schlüsse zu ziehen, entsteht falsches Wissen vor allem durch Veränderungen in der Umgebung, die nicht beobachtet werden.
Da die Wissensbasis vor allem durch Bobachtungen gepflegt wird, können Veränderungen die nicht beobachtet werden, fakten der Wissensbasis obsolet machen.

Das bewusste Modellieren von kognitiver Verzerrung ließe sich sowohl durch beeinflussung der Wünsche als auch des Wissens umsetzen. 
Das festsetzen des Grundwissens eines Agenten könnte bereits falsche Fakten berücksichtigen. 
Solche falschen Fakten müssten die Vorhaben und die resultierenden Aktionen beeinflussen.
Es ist allerdings möglich das der Agent seine Wissensbasis durch neue Beobachtungen anpasst und das Grundwissen nach gewisser Zeit "repariert".
Eine Verhinderung dieser Selbstheilung könnte durch Einschränkung der Beobachtungen weiter gehemmt werden.
TODO: Verzerrung durch Wünsche
% bias durch beliefs zb? schwarz = kriminell etc\\
% bias durch intentions? bevorzugen von bestimmten zielen\\
% interaktionen mit anderen Agenten sind nicht eingeplant
\subsection{BDI Erweiterungen}
Emotionen -> natürliche quelle für verzerrungen\\
wo erwähnt?\\
keine genaue spezifikation\\
\\
Verpflichtungen -> moddelieren von sozialem "druck"\\
formale beschreibung -> keine implementierung vorhanden\\

