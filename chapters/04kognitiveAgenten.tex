% !TEX root = ../main.tex
\chapter{Kognitive Agenten}
Eine weitere Klasse von agenten sind Kognitive Agenten.
Sie sind von der kognitiven Forschung beeinflusst und berücksichtigen dementsprechend besonders den Prozess der Entscheidungsfindung.
Die Art und Weise wie diese Entscheidungen getroffen werden orientiert sich wieder am Menschen.
Dementsprechend eignen sich auch diese Modelle gut dazu soziales Verhalten zu simulieren and zu analysieren.

Eines der Modelle - das Consumat Modell - soll im folgenden Abschnitt, als ein Vertreter dieser Klasse, genauer betrachtet werden.

\section{Consumat Modell}
Das Consumat Modell wurde von Jager und Janssen entwickelt um das verhalten von Konsumenten am Markt zu modellieren\cite{}.%janssen jager 2001

In dem Modell wird davon ausgegangen das ein Agent verschiedenen Bedürfnissen unterliegt, die er zu befriedigen versucht.
Dabei werden drei zentrale Annahmen gemacht:
\begin{itemize}
    \item Bedürfnisse sind mehrdimensional
    \item Der Entscheidungsprozess benötigt sowohl kognitiven als auch zeitlichen Aufwand
    \item Entscheidungen werden unter einem varierenden Grad der Unsicherheit getroffen
\end{itemize}
Außerdem wird davon ausgegangen, dass sich Bedürfnisse nicht nur abschwechen, wenn sie befriedigt werden, sondern auch mit dem verstreichen von Zeit.

Als Vorlage für die modellierten Bedürfnisse wurde die Arbeit Maslow\cite{} und Max-Neef genommen.
Dort werden neun menschliche bedürfnisse unterschieden, die für das Consumat Modell auf persöhnliche-, soziale- und Statusbedürfnisse reduziert wurden.
Dabei können die Arten auch untereinander im Konflikt stehen.

Dadurch dass der Entscheidungsprozess Kosten unterliegt, werden verschiedene Heuristiken eingeführt, die diesen Vorgang vereinfachen können.
Somit werden unter den entsprechenden Umständen nicht nur die Resultate von Entscheidungen optimiert, sondern auch der Entscheidungsprozess selbst.

Die Wahl der Heuristik ist dabei Abhängig von der Ausprägung der Bedürfnisse als auch von der Unsicherheit der Resultate.
Bei erhöhter Unsicherheit werden Entscheidungen an anderen Agenten orientiert und somit eine gewisse soziale Komponente integriert.
Außerdem korreliert die Stärke eines Bedürfnisses mit dem kognitiven Aufwand der für die Lösung aufgewendet wird.
Bei starken Bedürfnissen wird z.B. versucht das Ergebnis jeder möglichen Entscheidung zu berücksichtigen.
Während bei einer geringen Ausprägung nur die nächst bessere Möglichkeit übernommen wird.
\\
ausbreitung von verzerrung durch abgucken bei anderen\\

% PECS als alternative zu BDI entwickelt\cite{}\\%schmidt2002b
% deutlich komplexere struktur als BDI\\
% basiert auf adam model\cite{}\\%schmidt2002
% versucht körperliche, emotionale, kognitive und soziale einflusse für entscheidungsfindung zu berücksichtigen\\
% konzeptuelles modell, keine referenz implementierung vorhanden\\